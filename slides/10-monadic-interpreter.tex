%% -*- coding: utf-8 -*-
\documentclass{beamer}

%% -*- coding: utf-8 -*-
\usetheme{Boadilla} % default
\useoutertheme{infolines}
\setbeamertemplate{navigation symbols}{} 

\usepackage{etex}
\usepackage{alltt}
\usepackage{pifont}
\usepackage{color}
\usepackage[utf8]{inputenc}
%\usepackage{german}
\usepackage{listings}
\lstset{language=Haskell}
\lstset{sensitive=true}
\usepackage{hyperref}
\hypersetup{colorlinks=true}
\usepackage[final]{pdfpages}
\usepackage{url}
\usepackage{arydshln} % dashed lines
\usepackage{tikz}
\usepackage{mathpartir}

\DeclareUnicodeCharacter{3BB}{\ensuremath{\lambda}}


\newcommand\cmark{\ding{51}}
\newcommand\xmark{\ding{55}}

\newcommand{\nat}{\mathbf{N}}

\usepackage[all]{xy}

%% new arrow tip for xy
\newdir{|>}{!/4.5pt/@{|}*:(1,-.2)@^{>}*:(1,+.2)@_{>}}

\newcommand\cid[1]{\textup{\textbf{#1}}} % class names
\newcommand\kw[1]{\textup{\textbf{#1}}}  % key words
\newcommand\tid[1]{\textup{\textsf{#1}}} % type names
\newcommand\vid[1]{\textup{\texttt{#1}}} % value names
\newcommand\Mid[1]{\textup{\texttt{#1}}} % method names

\newcommand\TODO[1][]{{\color{red}{\textbf{TODO: #1}}}}

\newcommand\String[1]{\texttt{\dq{}#1\dq{}}}

\newcommand\ClassHead[1]{%
  \ensuremath{\begin{array}{|l|}
      \hline
      \cid{#1}
      \\\hline
    \end{array}}}
\newcommand\AbstractClass[2]{%
  \ensuremath{\begin{array}{|l|}
      \hline
      \cid{\textit{#1}}
      \\\hline
      #2
      \hline
    \end{array}}}
\newcommand\Class[2]{%
  \ensuremath{\begin{array}{|l|}
      \hline
      \cid{#1}
      \\\hline
      #2
      \hline
    \end{array}}}
\newcommand\Attribute[3][black]{\textcolor{#1}{\Param{#2}{#3}}\\}
\newcommand\Methods{\hline}
\newcommand\MethodSig[3]{\Mid{#2} (#3): \,\tid{#1}\\}
\newcommand\CtorSig[2]{\Mid{#1} (#2)\\}
\newcommand\AbstractMethodSig[3]{\Mid{\textit{#2}} (#3): \,\tid{#1}\\}
\newcommand\Param[2]{\vid{#2}:~\tid{#1}}

\lstset{%
  frame=single,
  xleftmargin=2pt,
  stepnumber=1,
  numbers=left,
  numbersep=5pt,
  numberstyle=\ttfamily\tiny\color[gray]{0.3},
  belowcaptionskip=\bigskipamount,
  captionpos=b,
  escapeinside={*'}{'*},
  language=java,
  tabsize=2,
  emphstyle={\bf},
  commentstyle=\mdseries\it,
  stringstyle=\mdseries\rmfamily,
  showspaces=false,
  showtabs=false,
  keywordstyle=\bfseries,
  columns=fullflexible,
  basicstyle=\footnotesize\CodeFont,
  showstringspaces=false,
  morecomment=[l]\%,
  rangeprefix=////,
  includerangemarker=false,
}

\newcommand\CodeFont{\sffamily}

\definecolor{lightred}{rgb}{0.8,0,0}
\definecolor{darkgreen}{rgb}{0,0.5,0}
\definecolor{darkblue}{rgb}{0,0,0.5}

\newcommand\highlight[1]{\textcolor{blue}{\emph{#1}}}
\newcommand\GenClass[2]{\cid{#1}\texttt{<}\cid{#2}\texttt{>}}

\newcommand\Colored[3]{\alt<#1>{\textcolor{#2}{#3}}{#3}}

\newcommand\nt[1]{\ensuremath{\langle#1\rangle}}

\newcommand{\free}{\operatorname{free}}
\newcommand{\bound}{\operatorname{bound}}
\newcommand{\var}{\operatorname{var}}
\newcommand\VSPBLS{\vspace{-\baselineskip}}

\newcommand\IF{\textit{IF}}
\newcommand\TRUE{\textit{TRUE}}
\newcommand\FALSE{\textit{FALSE}}

\newcommand\IFZ{\textit{IF0}}
\newcommand\ZERO{\textit{ZERO}}
\newcommand\SUCC{\textit{SUCC}}
\newcommand\ADD{\textit{ADD}}
\newcommand\SUB{\textit{SUB}}
\newcommand\MULT{\textit{MULT}}
\newcommand\DIV{\textit{DIV}}

\newcommand\PAIR{\textit{PAIR}}
\newcommand\FST{\textit{FST}}
\newcommand\SND{\textit{SND}}

\newcommand\CASE{\textit{CASE}}
\newcommand\LEFT{\textit{LEFT}}
\newcommand\RIGHT{\textit{RIGHT}}

\newcommand\Encode[1]{\lceil#1\rceil}
\newcommand\Reduce{\stackrel\ast\rightarrow_\beta}

\newcommand\Nat{\textit{Nat}}
\newcommand\Bool{\textit{Bool}}
\newcommand\Pair{\textit{Pair}}
\newcommand\Tfun[1]{#1\to}

\newcommand\Tenv{A}
\newcommand\Lam[1]{\lambda#1.}
\newcommand\App[1]{#1\,}
\newcommand\Succ{\textit{SUCC}\,}
\newcommand\Let[2]{\textit{let}\,#1=#2\,\textit{in}\,}

\newcommand\calE{\mathcal{E}}
\newcommand\calU{\mathcal{U}}
\newcommand\calP{\mathcal{P}}
\newcommand\calW{\mathcal{W}}

\newcommand\GEN{\textit{gen}}
\newcommand\EFV[1]{\textit{fv} (#1)}
\newcommand\Dom[1]{\textit{dom} (#1)}

%%% Local Variables: 
%%% mode: latex
%%% TeX-master: nil
%%% End: 

%%% frontmatter
%% -*- coding: utf-8 -*-

\title{Functional Programming}
\subtitle{Introduction}

\author[Peter Thiemann]{Prof. Dr. Peter Thiemann}
\institute[Univ. Freiburg]{Albert-Ludwigs-Universität Freiburg, Germany}
\date{SS 2019}


\subtitle
{Interpreters and Monads}

\begin{document}

\begin{frame}
  \titlepage
\end{frame}

%\begin{frame}
%  \frametitle{Outline}
%  \tableofcontents
  % You might wish to add the option [pausesections]
%\end{frame}


\begin{frame}[fragile]
  \frametitle{A simple expression language}
\begin{block}{Definition}
\begin{verbatim}
data Term  = Con Integer
           | Bin Term Op Term  
             deriving (Eq, Show)
           
data Op    = Add | Sub | Mul | Div
             deriving (Eq, Show)
\end{verbatim}  
\end{block}
\end{frame}             

\begin{frame}[fragile]
  \frametitle{A simple interpreter}
\begin{block}{Evaluation}
\begin{verbatim}
eval              :: Term -> Integer
eval (Con n)      =  n
eval (Bin t op u) =  sys op (eval t) (eval u)

sys Add       =  (+)         
sys Sub       =  (-)
sys Mul       =  (*)         
sys Div       =  div
\end{verbatim}  
\end{block}
\end{frame}             

\begin{frame}[fragile]
  \frametitle{Extending the interpreter}
  \begin{block}{Possible extensions}
    \begin{itemize}
    \item Error handling
    \item Counting evaluation steps
    \item Variables, state
    \item Output
    \end{itemize}
    \dots{} but without changing the structure of the interpreter!
  \end{block}
\end{frame}

\begin{frame}[fragile]
  \frametitle{Interpreter with error handling}
  \begin{block}{Exception}
\begin{verbatim}
data Exception a =  Raise  String
                 |  Return a

eval              :: Term -> Exception Integer
eval (Con n)      = Return n
eval (Bin t op u) = case eval t of
                     Raise  s -> Raise s
                     Return v -> case eval u of
                      Raise  s -> Raise s
                      Return w ->
                        if (op == Div && w == 0)
                        then  
                          Raise "div by zero"
                        else                                  
                          Return (sys op v w)
\end{verbatim}
\end{block}     
\end{frame}             


\begin{frame}[fragile]
  \frametitle{Monads to the rescue!}
\begin{block}{The type class Monad}
\begin{verbatim}
class Monad m where
  (>>=)  :: m a -> (a -> m b) -> m b
  return :: a -> m a
  fail   :: String -> m a
\end{verbatim}
  Here, \texttt{m} is a variable that can stand for \texttt{IO}, \texttt{Gen}, and other \textbf{type constructors}.
\end{block}
\end{frame}     


\begin{frame}[fragile]
  \frametitle{Monadic evaluator}
  \begin{alertblock}{The identity monad}
\begin{verbatim}
newtype Id a = Id a

instance Monad Id where
    return x = Id x
    x >>= f  = let Id y = x in f y
\end{verbatim}  
\end{alertblock}

\begin{exampleblock}{Monadic interpreter}
\begin{verbatim}
eval              :: Term -> Id Integer
eval (Con n)      =  return n
eval (Bin t op u) =  eval t  >>= \v ->
                     eval u  >>= \w ->
                     return (sys op v w)
\end{verbatim}  
\end{exampleblock}
\end{frame}             


\begin{frame}[fragile]
  \frametitle{Monadic interpreter with error handling}
  \begin{alertblock}{Exeception}
\begin{verbatim}
instance Monad Exception where
  return a = Return a
  m >>= f  = case m of 
               Raise  s -> Raise s
               Return v -> f v
  fail s   = Raise s
\end{verbatim}  
\end{alertblock}

\begin{exampleblock}{Interpreter}
\begin{verbatim}
eval             :: Term -> Exception Integer
eval (Con n)      = return n
eval (Bin t op u) = eval t  >>= \v ->
                    eval u  >>= \w ->
                    if (op == Div && w == 0) 
                     then fail "div by zero"
                    else return (sys op v w)
\end{verbatim}  
\end{exampleblock}
\end{frame}             

\begin{frame}[fragile]
  \frametitle{Interpreter with tracing}
  \begin{block}{Trace}
\begin{verbatim}
newtype Trace a = Trace (a, String)

eval :: Term -> Trace Integer
eval e@(Con n)      = Trace (n, trace e n)
eval e@(Bin t op u) = 
   let Trace (v, x) = eval t in
   let Trace (w, y) = eval u in
   let r = sys op v w in
   Trace (r, x ++ y ++ trace e r)

trace t n = "eval (" ++ show t ++ ") = "
              ++ show n ++ "\n"
\end{verbatim}  
\end{block}
\end{frame}


\begin{frame}[fragile]
  \frametitle{A monad for tracing}
  \begin{alertblock}{Trace}
\begin{verbatim}
instance Monad Trace where
  return a = (a, "")
  m >>= f  = let Trace (a, x) = m in
             let Trace (b, y) = f a in
             Trace (b, x ++ y)

output   :: String -> Trace ()
output s  = Trace ((), s)
\end{verbatim}  
\end{alertblock}
\end{frame}

\begin{frame}[fragile]
 \frametitle{Monadic interpreter with tracing}
 \begin{exampleblock}{Evaluation}
\begin{verbatim}
eval :: Term -> Trace Integer
eval e@(Con n) = output (trace e n) >>
                 return n
eval e@(Bin t op u) = eval t >>= \v ->
                      eval u >>= \w ->
                      let r = sys op v w in
                      output (trace e r) >>
                      return r
\end{verbatim}
\end{exampleblock}      
\end{frame}             

\begin{frame}[fragile]
  \frametitle{Interpreter with reduction count}\small
  \begin{block}{Count}
\begin{verbatim}
type Count a =  Int -> (a, Int)

eval              :: Term -> Count Integer
eval (Con n)      = \i -> (n, i)
eval (Bin t op u) = \i -> let (v, j) = eval t i in    
                          let (w, k) = eval u j in
                           (sys op v w, k + 1)
\end{verbatim} 
\end{block}
\end{frame}            


\begin{frame}[fragile]
  \frametitle{A monad for counting}
  \framesubtitle{The state monad}
  \begin{alertblock}{State}
\begin{verbatim}
data ST s a = ST (s -> (a, s))
exST (ST sas) = sas

instance Monad (ST s) where
  return a = ST (\s -> (a, s))
  m >>= f  = ST (\s -> let (a, s') = exST m s in
                       exST (f a) s')

type Count a = ST Int a

incr :: Count ()
incr =  ST (\i -> ((), i + 1))
\end{verbatim} 
\end{alertblock}
\end{frame}

\begin{frame}[fragile]
  \frametitle{Monadic interpreter with reduction count}
  \framesubtitle{Implementation}
\begin{exampleblock}{Evaluation}
\begin{verbatim}
eval :: Term -> Count Integer
eval (Con n)      =  return n
eval (Bin t op u) =  eval t >>= \v ->
                     eval u >>= \w ->
                     incr   >>
                     return (sys op v w)
\end{verbatim} 
\end{exampleblock}
\end{frame}            

\begin{frame}[fragile]
  \frametitle{Typical monads}
  \begin{block}{Already used}
  \begin{itemize}       
  \item Identity monad
  \item Exception monad
  \item State monad
  \item Writer monad
  \end{itemize} 
  \end{block}   
\end{frame}

\begin{frame}[fragile]
  \frametitle{Not every type constructor can be a monad}
  \framesubtitle{Monad laws}
  \begin{alertblock}{return is a left unit}
\begin{verbatim}
return x >>= f        == f x
\end{verbatim}
  \end{alertblock}
  \begin{alertblock}{return is a right unit}
\begin{verbatim}
m >>= return          == m
\end{verbatim}
  \end{alertblock}
  \begin{alertblock}{bind is associative}
\begin{verbatim}
m1 >>= \x -> (m2 >>= f) == (m1 >>= \x -> m2) >>= f
\end{verbatim}
  \end{alertblock}
\end{frame}

\begin{frame}[fragile]
  \frametitle{The Maybe monad}
  \begin{block}{More useful than you think}
                \begin{itemize}         
                        \item Computation that may or may not return a result
                        \item Database queries, dictionary operations, \dots
                \end{itemize}   
                \end{block}     
\pause
\begin{block}{Definition (predefined)}
\begin{verbatim}
data Maybe a = Nothing | Just a

instance Monad Maybe where
    return x       = Just x
    
    Nothing  >>= f = Nothing
    (Just x) >>= f = f x
\end{verbatim}  
\end{block}
\end{frame}             

\begin{frame}[fragile]
  \frametitle{The List monad}
  \begin{block}{Useful for}
                \begin{itemize}         
                        \item Handling multiple results
                        \item Backtracking
                \end{itemize}   
                \end{block}     
\pause
\begin{block}{Definition (predefined)}
\begin{verbatim}
instance Monad [] where
    return x = [x]
    m >>= f  = concatMap f m
\end{verbatim}  
\end{block}
where 
\begin{verbatim}
concatMap :: (a -> [b]) -> [a] -> [b]
concatMap = undefined
\end{verbatim}
\end{frame}             

\begin{frame}[fragile]
  \frametitle{The IO Monad}
  \begin{block}{Required for}
  \begin{itemize}
        \item any kind of I/O
        \item side effecting operation
        \item implementation is machine dependent
  \end{itemize}
  \end{block}
\end{frame}

\begin{frame}
  \frametitle{Challenges}
  \begin{itemize}[<+->]
  \item what if there are multiple effects?
  \item need to combine monads
  \item sequence matters (e.g., exception and state)
  \item some monads do not combine at all
  \item BUT we can go for something weaker
  \end{itemize}
  
\end{frame}


\end{document}


