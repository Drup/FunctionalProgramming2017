%% -*- coding: utf-8 -*-
\documentclass{beamer}

%% -*- coding: utf-8 -*-
\usetheme{Boadilla} % default
\useoutertheme{infolines}
\setbeamertemplate{navigation symbols}{} 

\usepackage{etex}
\usepackage{alltt}
\usepackage{pifont}
\usepackage{color}
\usepackage[utf8]{inputenc}
%\usepackage{german}
\usepackage{listings}
\lstset{language=Haskell}
\lstset{sensitive=true}
\usepackage{hyperref}
\hypersetup{colorlinks=true}
\usepackage[final]{pdfpages}
\usepackage{url}
\usepackage{arydshln} % dashed lines
\usepackage{tikz}
\usepackage{mathpartir}

\DeclareUnicodeCharacter{3BB}{\ensuremath{\lambda}}


\newcommand\cmark{\ding{51}}
\newcommand\xmark{\ding{55}}

\newcommand{\nat}{\mathbf{N}}

\usepackage[all]{xy}

%% new arrow tip for xy
\newdir{|>}{!/4.5pt/@{|}*:(1,-.2)@^{>}*:(1,+.2)@_{>}}

\newcommand\cid[1]{\textup{\textbf{#1}}} % class names
\newcommand\kw[1]{\textup{\textbf{#1}}}  % key words
\newcommand\tid[1]{\textup{\textsf{#1}}} % type names
\newcommand\vid[1]{\textup{\texttt{#1}}} % value names
\newcommand\Mid[1]{\textup{\texttt{#1}}} % method names

\newcommand\TODO[1][]{{\color{red}{\textbf{TODO: #1}}}}

\newcommand\String[1]{\texttt{\dq{}#1\dq{}}}

\newcommand\ClassHead[1]{%
  \ensuremath{\begin{array}{|l|}
      \hline
      \cid{#1}
      \\\hline
    \end{array}}}
\newcommand\AbstractClass[2]{%
  \ensuremath{\begin{array}{|l|}
      \hline
      \cid{\textit{#1}}
      \\\hline
      #2
      \hline
    \end{array}}}
\newcommand\Class[2]{%
  \ensuremath{\begin{array}{|l|}
      \hline
      \cid{#1}
      \\\hline
      #2
      \hline
    \end{array}}}
\newcommand\Attribute[3][black]{\textcolor{#1}{\Param{#2}{#3}}\\}
\newcommand\Methods{\hline}
\newcommand\MethodSig[3]{\Mid{#2} (#3): \,\tid{#1}\\}
\newcommand\CtorSig[2]{\Mid{#1} (#2)\\}
\newcommand\AbstractMethodSig[3]{\Mid{\textit{#2}} (#3): \,\tid{#1}\\}
\newcommand\Param[2]{\vid{#2}:~\tid{#1}}

\lstset{%
  frame=single,
  xleftmargin=2pt,
  stepnumber=1,
  numbers=left,
  numbersep=5pt,
  numberstyle=\ttfamily\tiny\color[gray]{0.3},
  belowcaptionskip=\bigskipamount,
  captionpos=b,
  escapeinside={*'}{'*},
  language=java,
  tabsize=2,
  emphstyle={\bf},
  commentstyle=\mdseries\it,
  stringstyle=\mdseries\rmfamily,
  showspaces=false,
  showtabs=false,
  keywordstyle=\bfseries,
  columns=fullflexible,
  basicstyle=\footnotesize\CodeFont,
  showstringspaces=false,
  morecomment=[l]\%,
  rangeprefix=////,
  includerangemarker=false,
}

\newcommand\CodeFont{\sffamily}

\definecolor{lightred}{rgb}{0.8,0,0}
\definecolor{darkgreen}{rgb}{0,0.5,0}
\definecolor{darkblue}{rgb}{0,0,0.5}

\newcommand\highlight[1]{\textcolor{blue}{\emph{#1}}}
\newcommand\GenClass[2]{\cid{#1}\texttt{<}\cid{#2}\texttt{>}}

\newcommand\Colored[3]{\alt<#1>{\textcolor{#2}{#3}}{#3}}

\newcommand\nt[1]{\ensuremath{\langle#1\rangle}}

\newcommand{\free}{\operatorname{free}}
\newcommand{\bound}{\operatorname{bound}}
\newcommand{\var}{\operatorname{var}}
\newcommand\VSPBLS{\vspace{-\baselineskip}}

\newcommand\IF{\textit{IF}}
\newcommand\TRUE{\textit{TRUE}}
\newcommand\FALSE{\textit{FALSE}}

\newcommand\IFZ{\textit{IF0}}
\newcommand\ZERO{\textit{ZERO}}
\newcommand\SUCC{\textit{SUCC}}
\newcommand\ADD{\textit{ADD}}
\newcommand\SUB{\textit{SUB}}
\newcommand\MULT{\textit{MULT}}
\newcommand\DIV{\textit{DIV}}

\newcommand\PAIR{\textit{PAIR}}
\newcommand\FST{\textit{FST}}
\newcommand\SND{\textit{SND}}

\newcommand\CASE{\textit{CASE}}
\newcommand\LEFT{\textit{LEFT}}
\newcommand\RIGHT{\textit{RIGHT}}

\newcommand\Encode[1]{\lceil#1\rceil}
\newcommand\Reduce{\stackrel\ast\rightarrow_\beta}

\newcommand\Nat{\textit{Nat}}
\newcommand\Bool{\textit{Bool}}
\newcommand\Pair{\textit{Pair}}
\newcommand\Tfun[1]{#1\to}

\newcommand\Tenv{A}
\newcommand\Lam[1]{\lambda#1.}
\newcommand\App[1]{#1\,}
\newcommand\Succ{\textit{SUCC}\,}
\newcommand\Let[2]{\textit{let}\,#1=#2\,\textit{in}\,}

\newcommand\calE{\mathcal{E}}
\newcommand\calU{\mathcal{U}}
\newcommand\calP{\mathcal{P}}
\newcommand\calW{\mathcal{W}}

\newcommand\GEN{\textit{gen}}
\newcommand\EFV[1]{\textit{fv} (#1)}
\newcommand\Dom[1]{\textit{dom} (#1)}

%%% Local Variables: 
%%% mode: latex
%%% TeX-master: nil
%%% End: 

%%% frontmatter
%% -*- coding: utf-8 -*-

\title{Functional Programming}
\subtitle{Introduction}

\author[Peter Thiemann]{Prof. Dr. Peter Thiemann}
\institute[Univ. Freiburg]{Albert-Ludwigs-Universität Freiburg, Germany}
\date{SS 2019}


\subtitle{More about lists}
\usepackage{tikz}


\begin{document}

\begin{frame}
  \titlepage
\end{frame}
%----------------------------------------------------------------------
\begin{frame}[fragile]
  \frametitle{Lists recap}
  \begin{block}{Zero or more values}
\begin{verbatim}
[]   [1]   [True, False]    ["a", "bunch", "of", "flowers"]
\end{verbatim}
  \end{block}
  \begin{block}{All have the same type}
\begin{verbatim}
[True, False] -- good
[1, "two", False] -- bad, type error
\end{verbatim}
  \end{block}
  \begin{block}{Order matters}
\begin{verbatim}
[1,2,3] /= [3,2,1]
\end{verbatim}
  \end{block}
\end{frame}
%----------------------------------------------------------------------
\begin{frame}[fragile]
  \frametitle{List syntax}
\begin{verbatim}
(1 : (2 : (3 : []))) 
==
1 : 2 : 3 : [] 
==
[1,2,3]
\end{verbatim}
  Strings are lists of characters
\begin{verbatim}
"Hearts" == ['H','e','a','r','t','s']
\end{verbatim}
\end{frame}
%----------------------------------------------------------------------
\begin{frame}[fragile]
  \frametitle{Defining a list datatype}
    \begin{block}<+->{The values of type [a] are \dots}
    \begin{itemize}
    \item either \texttt{[]}, the empty list
    \item or \texttt{x:xs} where \texttt{x} has type \texttt{a} and
      \texttt{xs} has type \texttt{[a]} \\
      ``\texttt{:}'' is pronounced ``cons''
    \end{itemize}
  \end{block}
\begin{verbatim}
data List a = ...
\end{verbatim}
  \begin{block}<+->{Corresponding definition}
\begin{verbatim}
data List a = Nil | Cons a (List a)
\end{verbatim}
    \begin{itemize}
    \item New: \texttt{List} is a parametric datatype with type
      parameter \texttt{a}
    \item Many functions on lists are also parametric (i.e., \textbf{polymorphic}) 
  \end{itemize}
  \end{block}  
\end{frame}
%----------------------------------------------------------------------
\begin{frame}[fragile]
  \frametitle{Polymorphic functions on lists}
\begin{verbatim}
length :: [a] -> Int
(++)   :: [a] -> [a] -> [a]
concat :: [[a]] -> [a]
take   :: Int -> [a] -> [a]
zip    :: [a] -> [b] -> [(a,b)]

map    :: (a -> b)    -> [a] -> [b]
filter :: (a -> Bool) -> [a] -> [a]
\end{verbatim}
\end{frame}
%----------------------------------------------------------------------
\begin{frame}[fragile]
  \frametitle{Prelude functions on lists}
  \begin{block}<+->{Functions on specific lists}
\begin{verbatim}
and, or          :: [Bool] -> Bool
words, lines     :: String -> [String]
unwords, unlines :: [String] -> String
\end{verbatim}
  \end{block}
  \begin{block}<+->{Overloaded functions on lists}
\begin{verbatim}
sum, product     :: Num a => [a] -> a
elem             :: Eq a => a -> [a] -> Bool
sort             :: Ord a => [a] -> [a]
\end{verbatim}
  \end{block}
\end{frame}
%----------------------------------------------------------------------
%----------------------------------------------------------------------
\begin{frame}[fragile]
  \frametitle{Some examples \dots}
  \begin{itemize}
  \item append, reverse
  \item sum, product
  \item take, drop, splitAt
  \item zip, unzip
  \item insert, isort, qsort
  \item QuickCheck: collect, classify
  \end{itemize}
\end{frame}
%----------------------------------------------------------------------
\begin{frame}[fragile]
  \frametitle{Quicksort!}
\begin{verbatim}
qsort :: Ord a => [a] -> [a]
qsort [] = []
qsort (x:xs) = qsort smaller ++ [x] ++ qsort bigger
  where
    smaller = filter (<= x) xs
    bigger  = filter (> x) xs
\end{verbatim}
\end{frame}
%----------------------------------------------------------------------
\begin{frame}[fragile]
  \frametitle{An unfortunate QuickCheck --- ghci interaction}
  \begin{block}<+->{Two properties}
\begin{verbatim}
prop_take_drop n xs = take n xs ++ drop n xs == xs
nonprop_take_drop n xs = drop n xs ++ take n xs == xs
\end{verbatim}
  \end{block}
  \begin{block}<+->{Testing  \dots}
\begin{verbatim}
*Main> quickCheck prop_take_drop
+++ OK, passed 100 tests.
*Main> quickCheck nonprop_take_drop
+++ OK, passed 100 tests.
\end{verbatim}
  \end{block}
  \begin{block}<+->{Oops! what went wrong?}
\begin{verbatim}
prop_take_drop :: Eq a => Int -> [a] -> Bool
nonprop_take_drop :: Eq a => Int -> [a] -> Bool
\end{verbatim}
    \vspace{-\baselineskip}
    \begin{itemize}
    \item The properties have polymorphic types, but\dots
    \item QuickCheck does not work with polymorphic types!
    \end{itemize}
  \end{block}
\end{frame}
%----------------------------------------------------------------------
\begin{frame}[fragile]
  \frametitle{Ghci ``helps''}
  \begin{itemize}
  \item Instead of indicating the problem, ghci chooses a more specific \textbf{default type} 
  \item In this case, it plugs the unit type for \texttt{a}
  \item QuickCheck tests
\begin{verbatim}
prop_take_drop :: Eq a => Int -> [()] -> Bool
nonprop_take_drop :: Eq a => Int -> [()] -> Bool
\end{verbatim}
  \item Order does not matter when all elements are the same\dots
  \end{itemize}
\end{frame}
%----------------------------------------------------------------------
\begin{frame}[fragile]
  \frametitle{Force ghci to be unhelpful}
  \begin{itemize}
  \item Use type signatures
  \item Disable defaulting
\begin{verbatim}
*Main> :set -XNoExtendedDefaultRules
\end{verbatim}
  \item Restrict types used in defaulting
\begin{verbatim}
*Main> default (Integer, Double)
\end{verbatim}
  \end{itemize}
\end{frame}
%----------------------------------------------------------------------

\begin{frame}
  \frametitle{Break Time --- Questions?}
  \begin{center}
    \tikz{\node[scale=15] at (0,0){?};}
  \end{center}
\end{frame}


\end{document}

%%% Local Variables: 
%%% mode: latex
%%% TeX-master: t
%%% End: 
