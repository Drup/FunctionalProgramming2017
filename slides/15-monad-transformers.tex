%% -*- coding: utf-8 -*-
\documentclass[pdftex,aspectratio=169]{beamer}

\usepackage[T1]{fontenc}
\usepackage[utf8]{inputenc}
\usepackage{beramono}

%% -*- coding: utf-8 -*-
\usetheme{Boadilla} % default
\useoutertheme{infolines}
\setbeamertemplate{navigation symbols}{} 

\usepackage{etex}
\usepackage{alltt}
\usepackage{pifont}
\usepackage{color}
\usepackage[utf8]{inputenc}
%\usepackage{german}
\usepackage{listings}
\lstset{language=Haskell}
\lstset{sensitive=true}
\usepackage{hyperref}
\hypersetup{colorlinks=true}
\usepackage[final]{pdfpages}
\usepackage{url}
\usepackage{arydshln} % dashed lines
\usepackage{tikz}
\usepackage{mathpartir}

\DeclareUnicodeCharacter{3BB}{\ensuremath{\lambda}}


\newcommand\cmark{\ding{51}}
\newcommand\xmark{\ding{55}}

\newcommand{\nat}{\mathbf{N}}

\usepackage[all]{xy}

%% new arrow tip for xy
\newdir{|>}{!/4.5pt/@{|}*:(1,-.2)@^{>}*:(1,+.2)@_{>}}

\newcommand\cid[1]{\textup{\textbf{#1}}} % class names
\newcommand\kw[1]{\textup{\textbf{#1}}}  % key words
\newcommand\tid[1]{\textup{\textsf{#1}}} % type names
\newcommand\vid[1]{\textup{\texttt{#1}}} % value names
\newcommand\Mid[1]{\textup{\texttt{#1}}} % method names

\newcommand\TODO[1][]{{\color{red}{\textbf{TODO: #1}}}}

\newcommand\String[1]{\texttt{\dq{}#1\dq{}}}

\newcommand\ClassHead[1]{%
  \ensuremath{\begin{array}{|l|}
      \hline
      \cid{#1}
      \\\hline
    \end{array}}}
\newcommand\AbstractClass[2]{%
  \ensuremath{\begin{array}{|l|}
      \hline
      \cid{\textit{#1}}
      \\\hline
      #2
      \hline
    \end{array}}}
\newcommand\Class[2]{%
  \ensuremath{\begin{array}{|l|}
      \hline
      \cid{#1}
      \\\hline
      #2
      \hline
    \end{array}}}
\newcommand\Attribute[3][black]{\textcolor{#1}{\Param{#2}{#3}}\\}
\newcommand\Methods{\hline}
\newcommand\MethodSig[3]{\Mid{#2} (#3): \,\tid{#1}\\}
\newcommand\CtorSig[2]{\Mid{#1} (#2)\\}
\newcommand\AbstractMethodSig[3]{\Mid{\textit{#2}} (#3): \,\tid{#1}\\}
\newcommand\Param[2]{\vid{#2}:~\tid{#1}}

\lstset{%
  frame=single,
  xleftmargin=2pt,
  stepnumber=1,
  numbers=left,
  numbersep=5pt,
  numberstyle=\ttfamily\tiny\color[gray]{0.3},
  belowcaptionskip=\bigskipamount,
  captionpos=b,
  escapeinside={*'}{'*},
  language=java,
  tabsize=2,
  emphstyle={\bf},
  commentstyle=\mdseries\it,
  stringstyle=\mdseries\rmfamily,
  showspaces=false,
  showtabs=false,
  keywordstyle=\bfseries,
  columns=fullflexible,
  basicstyle=\footnotesize\CodeFont,
  showstringspaces=false,
  morecomment=[l]\%,
  rangeprefix=////,
  includerangemarker=false,
}

\newcommand\CodeFont{\sffamily}

\definecolor{lightred}{rgb}{0.8,0,0}
\definecolor{darkgreen}{rgb}{0,0.5,0}
\definecolor{darkblue}{rgb}{0,0,0.5}

\newcommand\highlight[1]{\textcolor{blue}{\emph{#1}}}
\newcommand\GenClass[2]{\cid{#1}\texttt{<}\cid{#2}\texttt{>}}

\newcommand\Colored[3]{\alt<#1>{\textcolor{#2}{#3}}{#3}}

\newcommand\nt[1]{\ensuremath{\langle#1\rangle}}

\newcommand{\free}{\operatorname{free}}
\newcommand{\bound}{\operatorname{bound}}
\newcommand{\var}{\operatorname{var}}
\newcommand\VSPBLS{\vspace{-\baselineskip}}

\newcommand\IF{\textit{IF}}
\newcommand\TRUE{\textit{TRUE}}
\newcommand\FALSE{\textit{FALSE}}

\newcommand\IFZ{\textit{IF0}}
\newcommand\ZERO{\textit{ZERO}}
\newcommand\SUCC{\textit{SUCC}}
\newcommand\ADD{\textit{ADD}}
\newcommand\SUB{\textit{SUB}}
\newcommand\MULT{\textit{MULT}}
\newcommand\DIV{\textit{DIV}}

\newcommand\PAIR{\textit{PAIR}}
\newcommand\FST{\textit{FST}}
\newcommand\SND{\textit{SND}}

\newcommand\CASE{\textit{CASE}}
\newcommand\LEFT{\textit{LEFT}}
\newcommand\RIGHT{\textit{RIGHT}}

\newcommand\Encode[1]{\lceil#1\rceil}
\newcommand\Reduce{\stackrel\ast\rightarrow_\beta}

\newcommand\Nat{\textit{Nat}}
\newcommand\Bool{\textit{Bool}}
\newcommand\Pair{\textit{Pair}}
\newcommand\Tfun[1]{#1\to}

\newcommand\Tenv{A}
\newcommand\Lam[1]{\lambda#1.}
\newcommand\App[1]{#1\,}
\newcommand\Succ{\textit{SUCC}\,}
\newcommand\Let[2]{\textit{let}\,#1=#2\,\textit{in}\,}

\newcommand\calE{\mathcal{E}}
\newcommand\calU{\mathcal{U}}
\newcommand\calP{\mathcal{P}}
\newcommand\calW{\mathcal{W}}

\newcommand\GEN{\textit{gen}}
\newcommand\EFV[1]{\textit{fv} (#1)}
\newcommand\Dom[1]{\textit{dom} (#1)}

%%% Local Variables: 
%%% mode: latex
%%% TeX-master: nil
%%% End: 

%%% frontmatter
%% -*- coding: utf-8 -*-

\title{Functional Programming}
\subtitle{Introduction}

\author[Peter Thiemann]{Prof. Dr. Peter Thiemann}
\institute[Univ. Freiburg]{Albert-Ludwigs-Universität Freiburg, Germany}
\date{SS 2019}


\author[Gabriel Radanne]{Dr. Gabriel Radanne}
\subtitle
{Monad Transformers}


\renewcommand\CodeFont{\ttfamily}
\lstset{%
  frame=none,
  language=Haskell
}

\begin{document}

\begin{frame}
  \titlepage
\end{frame}

\begin{frame}
  \frametitle{Reminder: Monad}
  \begin{block}{Definition of a Monad -- Lecture 7}
    \begin{itemize}
    \item abstract datatype for instructions that produce values
    \item built-in combination \lstinline{>>=}
    \item abstracts over different interpretations (computations)
    \end{itemize}
  \end{block}
\end{frame}

\begin{frame}[fragile]
  \frametitle{Monad definition}
  \begin{block}{The type class Monad}
\begin{lstlisting}
class Monad m where
  (>>=)  :: m a -> (a -> m b) -> m b
  return :: a -> m a
  fail   :: String -> m a
\end{lstlisting}

with the following laws:
\begin{itemize}
\item \lstinline{return x >>= f == f x}
\item \lstinline{m >>= return == m}
\item \lstinline{(m >>= f) >>= g == m >>= (\x -> f x >>= g)}
\end{itemize}

\end{block}
\end{frame}

\begin{frame}
  \frametitle{What about Composition?}
  \begin{itemize}[<+->]
  \item Applicatives compose (as seen in the lecture on parsing).
  \item Monads do not necessarily compose.
  \item We sometimes want to use multiples monads at once!
  \end{itemize}
\end{frame}

\begin{frame}
  \frametitle{Why combine monads}
  Lecture 10: Monadic interpreters.

  Interpreters can have many features:
  \begin{itemize}
  \item Failure (\lstinline{Maybe}).
  \item Keeping some state (\lstinline{State}).
  \item Reading from the environment (\lstinline{Reader}).
  \item \dots
  \end{itemize}

  To implement an interpreter, we need to combine all these monads!
\end{frame}

\begin{frame}[fragile]
  \frametitle{Let's combine Monads! -- State alone}

  \begin{block}{The State monad}
\begin{lstlisting}
data ST s a = ST (s -> (a, s))
runST (ST sas) = sas

instance Monad (ST s) where
  return a = ST (\s -> (a, s))
  m >>= f  = ST (\s ->
                     let (a, s') = runST m s in
                     runST (f a) s')
\end{lstlisting}
  \end{block}
\end{frame}

\begin{frame}[fragile]
  \frametitle{Let's combine Monads! -- Maybe+State}
  \begin{block}{The MaybeState monad}
      \begin{lstlisting}
data MaybeState s a = MS { runMS :: s -> Maybe (a, s) }

....

instance Monad (MST s) where
  return a = MST (\s -> Just (a, s))
  ms >>= f  = MST (\s -> case runMST ms s of
                          Nothing -> Nothing
                          Just (a,s') -> runMST (f a) s')
      \end{lstlisting}
    \end{block}
    \pause
    We would have to write this again for each combination!
\end{frame}

\begin{frame}[fragile]
  \frametitle{Alternative solution: Monad transformers}

  Monad transformers offer a better solution:
  \begin{block}{}
\begin{lstlisting}
class MonadTrans t where
  lift :: Monad m => m a -> t m a
\end{lstlisting}

\end{block}

A monad transformer \lstinline{t} takes a monad \lstinline{m} and yield a new monad
\lstinline{(t m)}.

\lstinline{lift} allows to lift a computation from the underlying monad to the new monad.
\end{frame}

\begin{frame}[fragile]
  \frametitle{\texttt{MaybeT}}
  \begin{block}{Definition}
\begin{lstlisting}
newtype MaybeT m a = MaybeT { runMaybeT :: m (Maybe a) }

instance (Monad m) => Monad (MaybeT m) where
  return = MaybeT . return . Just
  (MaybeT mmx) >>= f = MaybeT $ do
    mx <- mmx
    case mx of
      Nothing -> return Nothing
      Just x  -> runMaybeT (f x)

instance MonadTrans MaybeT where
  lift mx = MaybeT $ do { x <- mx ; return $ Just x }
\end{lstlisting}
  \end{block}
\end{frame}

\begin{frame}[fragile]
  \frametitle{A simple usage of MaybeT}
  We can recover the ``normal'' monad by applying to \lstinline{Identity}.
  \begin{block}{}
  \begin{lstlisting}
type MaybeLike = MaybeT Identity
  \end{lstlisting}
\end{block}
\end{frame}


\begin{frame}[fragile]
  \frametitle{\texttt{StateT}}
  \begin{block}{Definition}
\begin{lstlisting}
newtype StateT s m a = StateT { runStateT :: s -> m (a,s) }

instance (Monad m) => Monad (StateT m) where
  return a = StateT $ \s -> return (a, s)
  m >>= f  = StateT $ \s -> do
      (a, s') <- runStateT m s
      runStateT (f a) s'

instance MonadTrans StateT where
  lift ma = StateT $ \s -> do { a <- ma ; return (a, s) }
\end{lstlisting}
  \end{block}
\end{frame}


\begin{frame}[fragile]
  \frametitle{Let's combine Monads with transformers!}

  Demo!
\end{frame}


\begin{frame}[fragile]
  \frametitle{\texttt{ReaderT}}
  \begin{block}{Definition}
\begin{lstlisting}
newtype ReaderT r m a = ReaderT { runReaderT :: r -> m a }

ask :: (Monad m) => ReaderT r m r
ask = ReaderT return

instance Monad m => Monad (ReaderT r m) where
    return  = lift . return
    m >>= k = ReaderT $ \r -> do
                 a <- runReaderT m r
                 runReaderT (k a) r

instance MonadTrans (ReaderT r) where
    lift m = ReaderT (const m)
\end{lstlisting}
  \end{block}
\end{frame}

\begin{frame}[fragile]
  \frametitle{Back to interpreters}

  \begin{block}{During lecture 10, a monadic interpreter for:}
    \begin{lstlisting}
data Term  = Con Integer
           | Bin Term Op Term
             deriving (Eq, Show)

data Op    = Add | Sub | Mul | Div
             deriving (Eq, Show)
    \end{lstlisting}
  \end{block}
  \pause
  Different interpreters with various features:
  \begin{itemize}
  \item Failure
  \item Counting instructions
  \item Traces
  \end{itemize}
\end{frame}


\begin{frame}
  \frametitle{Key points}
  \begin{itemize}[<+->]
  \item Monads do not always compose ...
  \item But monad transformers help.
  \item Order is important!
  \item You should not overdo it.
  \item It's all in the \lstinline{mtl} library.
  \end{itemize}
\end{frame}

\end{document}




%%% Local Variables:
%%% mode: latex
%%% TeX-master: t
%%% End:
