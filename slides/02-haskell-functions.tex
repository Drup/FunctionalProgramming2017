%% -*- coding: utf-8 -*-
\documentclass{beamer}

%% -*- coding: utf-8 -*-
\usetheme{Boadilla} % default
\useoutertheme{infolines}
\setbeamertemplate{navigation symbols}{} 

\usepackage{etex}
\usepackage{alltt}
\usepackage{pifont}
\usepackage{color}
\usepackage[utf8]{inputenc}
%\usepackage{german}
\usepackage{listings}
\lstset{language=Haskell}
\lstset{sensitive=true}
\usepackage{hyperref}
\hypersetup{colorlinks=true}
\usepackage[final]{pdfpages}
\usepackage{url}
\usepackage{arydshln} % dashed lines
\usepackage{tikz}
\usepackage{mathpartir}

\DeclareUnicodeCharacter{3BB}{\ensuremath{\lambda}}


\newcommand\cmark{\ding{51}}
\newcommand\xmark{\ding{55}}

\newcommand{\nat}{\mathbf{N}}

\usepackage[all]{xy}

%% new arrow tip for xy
\newdir{|>}{!/4.5pt/@{|}*:(1,-.2)@^{>}*:(1,+.2)@_{>}}

\newcommand\cid[1]{\textup{\textbf{#1}}} % class names
\newcommand\kw[1]{\textup{\textbf{#1}}}  % key words
\newcommand\tid[1]{\textup{\textsf{#1}}} % type names
\newcommand\vid[1]{\textup{\texttt{#1}}} % value names
\newcommand\Mid[1]{\textup{\texttt{#1}}} % method names

\newcommand\TODO[1][]{{\color{red}{\textbf{TODO: #1}}}}

\newcommand\String[1]{\texttt{\dq{}#1\dq{}}}

\newcommand\ClassHead[1]{%
  \ensuremath{\begin{array}{|l|}
      \hline
      \cid{#1}
      \\\hline
    \end{array}}}
\newcommand\AbstractClass[2]{%
  \ensuremath{\begin{array}{|l|}
      \hline
      \cid{\textit{#1}}
      \\\hline
      #2
      \hline
    \end{array}}}
\newcommand\Class[2]{%
  \ensuremath{\begin{array}{|l|}
      \hline
      \cid{#1}
      \\\hline
      #2
      \hline
    \end{array}}}
\newcommand\Attribute[3][black]{\textcolor{#1}{\Param{#2}{#3}}\\}
\newcommand\Methods{\hline}
\newcommand\MethodSig[3]{\Mid{#2} (#3): \,\tid{#1}\\}
\newcommand\CtorSig[2]{\Mid{#1} (#2)\\}
\newcommand\AbstractMethodSig[3]{\Mid{\textit{#2}} (#3): \,\tid{#1}\\}
\newcommand\Param[2]{\vid{#2}:~\tid{#1}}

\lstset{%
  frame=single,
  xleftmargin=2pt,
  stepnumber=1,
  numbers=left,
  numbersep=5pt,
  numberstyle=\ttfamily\tiny\color[gray]{0.3},
  belowcaptionskip=\bigskipamount,
  captionpos=b,
  escapeinside={*'}{'*},
  language=java,
  tabsize=2,
  emphstyle={\bf},
  commentstyle=\mdseries\it,
  stringstyle=\mdseries\rmfamily,
  showspaces=false,
  showtabs=false,
  keywordstyle=\bfseries,
  columns=fullflexible,
  basicstyle=\footnotesize\CodeFont,
  showstringspaces=false,
  morecomment=[l]\%,
  rangeprefix=////,
  includerangemarker=false,
}

\newcommand\CodeFont{\sffamily}

\definecolor{lightred}{rgb}{0.8,0,0}
\definecolor{darkgreen}{rgb}{0,0.5,0}
\definecolor{darkblue}{rgb}{0,0,0.5}

\newcommand\highlight[1]{\textcolor{blue}{\emph{#1}}}
\newcommand\GenClass[2]{\cid{#1}\texttt{<}\cid{#2}\texttt{>}}

\newcommand\Colored[3]{\alt<#1>{\textcolor{#2}{#3}}{#3}}

\newcommand\nt[1]{\ensuremath{\langle#1\rangle}}

\newcommand{\free}{\operatorname{free}}
\newcommand{\bound}{\operatorname{bound}}
\newcommand{\var}{\operatorname{var}}
\newcommand\VSPBLS{\vspace{-\baselineskip}}

\newcommand\IF{\textit{IF}}
\newcommand\TRUE{\textit{TRUE}}
\newcommand\FALSE{\textit{FALSE}}

\newcommand\IFZ{\textit{IF0}}
\newcommand\ZERO{\textit{ZERO}}
\newcommand\SUCC{\textit{SUCC}}
\newcommand\ADD{\textit{ADD}}
\newcommand\SUB{\textit{SUB}}
\newcommand\MULT{\textit{MULT}}
\newcommand\DIV{\textit{DIV}}

\newcommand\PAIR{\textit{PAIR}}
\newcommand\FST{\textit{FST}}
\newcommand\SND{\textit{SND}}

\newcommand\CASE{\textit{CASE}}
\newcommand\LEFT{\textit{LEFT}}
\newcommand\RIGHT{\textit{RIGHT}}

\newcommand\Encode[1]{\lceil#1\rceil}
\newcommand\Reduce{\stackrel\ast\rightarrow_\beta}

\newcommand\Nat{\textit{Nat}}
\newcommand\Bool{\textit{Bool}}
\newcommand\Pair{\textit{Pair}}
\newcommand\Tfun[1]{#1\to}

\newcommand\Tenv{A}
\newcommand\Lam[1]{\lambda#1.}
\newcommand\App[1]{#1\,}
\newcommand\Succ{\textit{SUCC}\,}
\newcommand\Let[2]{\textit{let}\,#1=#2\,\textit{in}\,}

\newcommand\calE{\mathcal{E}}
\newcommand\calU{\mathcal{U}}
\newcommand\calP{\mathcal{P}}
\newcommand\calW{\mathcal{W}}

\newcommand\GEN{\textit{gen}}
\newcommand\EFV[1]{\textit{fv} (#1)}
\newcommand\Dom[1]{\textit{dom} (#1)}

%%% Local Variables: 
%%% mode: latex
%%% TeX-master: nil
%%% End: 

%%% frontmatter
%% -*- coding: utf-8 -*-

\title{Functional Programming}
\subtitle{Introduction}

\author[Peter Thiemann]{Prof. Dr. Peter Thiemann}
\institute[Univ. Freiburg]{Albert-Ludwigs-Universität Freiburg, Germany}
\date{SS 2019}


\subtitle{Functions}
\usepackage{tikz}


\begin{document}

\begin{frame}
  \titlepage
\end{frame}
%----------------------------------------------------------------------
\begin{frame}[fragile]
  \frametitle{Function definition by cases}
  \begin{block}<+->{Example: Absolute value}
    Find the absolute value of a number
    \begin{itemize}
    \item if \texttt{x} is positive, result is \texttt{x}
    \item if \texttt{x} is negative, result is \texttt{-x}
    \end{itemize}
  \end{block}
  \begin{block}<+->{Definition}
\begin{verbatim}
-- returns the absolute value of x
absolute :: Integer -> Integer
absolute x | x >= 0 = x
absolute x | x < 0  = - x
\end{verbatim}
  \end{block}
\end{frame}

\begin{frame}[fragile,fragile]
  \frametitle{Alternative styles of definition}
  
  \begin{block}{One equation}
\begin{verbatim}
absolute' x | x >= 0 = x
            | x < 0  = -x
\end{verbatim}
  \end{block}
  
  \begin{block}{Using if-then-else in an expression}
\begin{verbatim}
absolute'' x = if x >= 0 then x else -x
\end{verbatim}
  \end{block}
\end{frame}

\begin{frame}
  \frametitle{Recursion}
  Standard approach to define functions in functional languages
  (\textbf{no loops!})
  \begin{itemize}
  \item
    Reduce a problem (e.g., \texttt{power x n}) to a smaller
    problem of the same kind 
  \item Eventually reach a base case that can be solved immediately
  \item Build up solutions from smaller solutions
  \end{itemize}
\end{frame}

\begin{frame}[fragile]
  \frametitle{Example: power}
  Compute \verb|x^n| without using the built-in operator
\begin{verbatim}
-- compute x to n-th power
power x 0         = 1
power x n | n > 0 = x * power x (n - 1)
\end{verbatim}
\end{frame}

\begin{frame}
  \frametitle{Example: Counting intersections}
  \begin{block}<+->{Task}
    \begin{itemize}
    \item Consider $n$ non-parallel lines in the plane
    \item How often do these lines intersect (at most)? Call this
      number $I (n)$.
    \end{itemize}
  \end{block}
  \begin{block}<+->{Base case: $n=0$ (as simple as possible!)}
    \begin{itemize}
    \item<+-> Zero lines produce zero intersections: $I(0) = 0$
    \end{itemize}
  \end{block}
  \begin{block}<+->{Inductive case: $n>0$}
    \begin{itemize}
    \item<+-> One line can intersect with the remaining
      lines at most $n-1$ times.
    \item<+-> Remove this line. The remaining lines can intersect at
      most $I (n-1)$ times
    \item<+-> Combine the above to $I (n) =  I (n-1) + n-1$
    \end{itemize}
  \end{block}
\end{frame}
\begin{frame}[fragile]
  \frametitle{Definition}
  \begin{block}{Counting intersections}
\begin{verbatim}
-- max number of intersections of n lines
nisect :: Integer -> Integer
nisect 0   = 0
nisect n | n > 0 = nisect (n - 1) + n - 1
\end{verbatim}
  \end{block}
\end{frame}
%----------------------------------------------------------------------

\begin{frame}
  \frametitle{Questions?}
  \begin{center}
    \tikz{\node[scale=15] at (0,0){?};}
  \end{center}
\end{frame}


\end{document}

%%% Local Variables: 
%%% mode: latex
%%% TeX-master: t
%%% End: 
