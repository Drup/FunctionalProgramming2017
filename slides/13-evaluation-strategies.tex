%% -*- coding: utf-8 -*-
\documentclass[pdftex,aspectratio=169]{beamer}

%% -*- coding: utf-8 -*-
\usetheme{Boadilla} % default
\useoutertheme{infolines}
\setbeamertemplate{navigation symbols}{} 

\usepackage{etex}
\usepackage{alltt}
\usepackage{pifont}
\usepackage{color}
\usepackage[utf8]{inputenc}
%\usepackage{german}
\usepackage{listings}
\lstset{language=Haskell}
\lstset{sensitive=true}
\usepackage{hyperref}
\hypersetup{colorlinks=true}
\usepackage[final]{pdfpages}
\usepackage{url}
\usepackage{arydshln} % dashed lines
\usepackage{tikz}
\usepackage{mathpartir}

\DeclareUnicodeCharacter{3BB}{\ensuremath{\lambda}}


\newcommand\cmark{\ding{51}}
\newcommand\xmark{\ding{55}}

\newcommand{\nat}{\mathbf{N}}

\usepackage[all]{xy}

%% new arrow tip for xy
\newdir{|>}{!/4.5pt/@{|}*:(1,-.2)@^{>}*:(1,+.2)@_{>}}

\newcommand\cid[1]{\textup{\textbf{#1}}} % class names
\newcommand\kw[1]{\textup{\textbf{#1}}}  % key words
\newcommand\tid[1]{\textup{\textsf{#1}}} % type names
\newcommand\vid[1]{\textup{\texttt{#1}}} % value names
\newcommand\Mid[1]{\textup{\texttt{#1}}} % method names

\newcommand\TODO[1][]{{\color{red}{\textbf{TODO: #1}}}}

\newcommand\String[1]{\texttt{\dq{}#1\dq{}}}

\newcommand\ClassHead[1]{%
  \ensuremath{\begin{array}{|l|}
      \hline
      \cid{#1}
      \\\hline
    \end{array}}}
\newcommand\AbstractClass[2]{%
  \ensuremath{\begin{array}{|l|}
      \hline
      \cid{\textit{#1}}
      \\\hline
      #2
      \hline
    \end{array}}}
\newcommand\Class[2]{%
  \ensuremath{\begin{array}{|l|}
      \hline
      \cid{#1}
      \\\hline
      #2
      \hline
    \end{array}}}
\newcommand\Attribute[3][black]{\textcolor{#1}{\Param{#2}{#3}}\\}
\newcommand\Methods{\hline}
\newcommand\MethodSig[3]{\Mid{#2} (#3): \,\tid{#1}\\}
\newcommand\CtorSig[2]{\Mid{#1} (#2)\\}
\newcommand\AbstractMethodSig[3]{\Mid{\textit{#2}} (#3): \,\tid{#1}\\}
\newcommand\Param[2]{\vid{#2}:~\tid{#1}}

\lstset{%
  frame=single,
  xleftmargin=2pt,
  stepnumber=1,
  numbers=left,
  numbersep=5pt,
  numberstyle=\ttfamily\tiny\color[gray]{0.3},
  belowcaptionskip=\bigskipamount,
  captionpos=b,
  escapeinside={*'}{'*},
  language=java,
  tabsize=2,
  emphstyle={\bf},
  commentstyle=\mdseries\it,
  stringstyle=\mdseries\rmfamily,
  showspaces=false,
  showtabs=false,
  keywordstyle=\bfseries,
  columns=fullflexible,
  basicstyle=\footnotesize\CodeFont,
  showstringspaces=false,
  morecomment=[l]\%,
  rangeprefix=////,
  includerangemarker=false,
}

\newcommand\CodeFont{\sffamily}

\definecolor{lightred}{rgb}{0.8,0,0}
\definecolor{darkgreen}{rgb}{0,0.5,0}
\definecolor{darkblue}{rgb}{0,0,0.5}

\newcommand\highlight[1]{\textcolor{blue}{\emph{#1}}}
\newcommand\GenClass[2]{\cid{#1}\texttt{<}\cid{#2}\texttt{>}}

\newcommand\Colored[3]{\alt<#1>{\textcolor{#2}{#3}}{#3}}

\newcommand\nt[1]{\ensuremath{\langle#1\rangle}}

\newcommand{\free}{\operatorname{free}}
\newcommand{\bound}{\operatorname{bound}}
\newcommand{\var}{\operatorname{var}}
\newcommand\VSPBLS{\vspace{-\baselineskip}}

\newcommand\IF{\textit{IF}}
\newcommand\TRUE{\textit{TRUE}}
\newcommand\FALSE{\textit{FALSE}}

\newcommand\IFZ{\textit{IF0}}
\newcommand\ZERO{\textit{ZERO}}
\newcommand\SUCC{\textit{SUCC}}
\newcommand\ADD{\textit{ADD}}
\newcommand\SUB{\textit{SUB}}
\newcommand\MULT{\textit{MULT}}
\newcommand\DIV{\textit{DIV}}

\newcommand\PAIR{\textit{PAIR}}
\newcommand\FST{\textit{FST}}
\newcommand\SND{\textit{SND}}

\newcommand\CASE{\textit{CASE}}
\newcommand\LEFT{\textit{LEFT}}
\newcommand\RIGHT{\textit{RIGHT}}

\newcommand\Encode[1]{\lceil#1\rceil}
\newcommand\Reduce{\stackrel\ast\rightarrow_\beta}

\newcommand\Nat{\textit{Nat}}
\newcommand\Bool{\textit{Bool}}
\newcommand\Pair{\textit{Pair}}
\newcommand\Tfun[1]{#1\to}

\newcommand\Tenv{A}
\newcommand\Lam[1]{\lambda#1.}
\newcommand\App[1]{#1\,}
\newcommand\Succ{\textit{SUCC}\,}
\newcommand\Let[2]{\textit{let}\,#1=#2\,\textit{in}\,}

\newcommand\calE{\mathcal{E}}
\newcommand\calU{\mathcal{U}}
\newcommand\calP{\mathcal{P}}
\newcommand\calW{\mathcal{W}}

\newcommand\GEN{\textit{gen}}
\newcommand\EFV[1]{\textit{fv} (#1)}
\newcommand\Dom[1]{\textit{dom} (#1)}

%%% Local Variables: 
%%% mode: latex
%%% TeX-master: nil
%%% End: 

%%% frontmatter
%% -*- coding: utf-8 -*-

\title{Functional Programming}
\subtitle{Introduction}

\author[Peter Thiemann]{Prof. Dr. Peter Thiemann}
\institute[Univ. Freiburg]{Albert-Ludwigs-Universität Freiburg, Germany}
\date{SS 2019}


\subtitle
{Evaluation and Typing}

\begin{document}

\begin{frame}
  \titlepage
\end{frame}

%\begin{frame}
%  \frametitle{Outline}
%  \tableofcontents
  % You might wish to add the option [pausesections]
%\end{frame}


\begin{frame}[fragile]
  \frametitle{Results of computations}
\begin{block}<+->{Normal forms}
  \begin{itemize}
  \item expensive to compute
  \item rarely needed for evaluation
  \item key to expense: evaluation under lambda
  \end{itemize}
\end{block}
\begin{block}<+->{Definition: Weak head-normal form}
  A pure lambda term is in \textbf{weak head-normal form} (or a \textbf{value}) iff it
  has the form  
  \begin{displaymath}
    V ::= \lambda x.M
  \end{displaymath}
  All other terms are \textbf{non-values}.
\end{block}
\end{frame}             

\begin{frame}[fragile]
  \frametitle{Deterministic Evaluation}
  \begin{block}<+->{Definition}
    A \textbf{reduction strategy} is a function that assigns to each closed, non-value lambda term the position of the next reduction. 
  \end{block}
  \begin{block}<+->{Reduction strategy: Call-by-name (related to Haskell)}
    \begin{displaymath}
    \begin{array}{c}
      (\lambda x.M)~N \rightarrow_{\beta} M[x \mapsto N]
    \end{array}
    \qquad
    \begin{array}{c}
      M \rightarrow_\beta M'
      \\\hline
      (M~N) \rightarrow_\beta (M'~N)
    \end{array}
  \end{displaymath}
\end{block}
\begin{block}<+->{Reduction strategy: Call-by-value (based on $\beta$-value reduction)}
  \begin{displaymath}
    \begin{array}{c}
      (\lambda x.M)~\textcolor{red}{V} \rightarrow_{\beta V} M[x \mapsto \textcolor{red}{V}]
    \end{array}
    \qquad
    \begin{array}{c}
      M \rightarrow_{\beta V} M'
      \\\hline
      (M~N) \rightarrow_{\beta V} (M'~N)
    \end{array}
    \qquad
    \begin{array}{c}
      N \rightarrow_{\beta V} N'
      \\\hline
      (\textcolor{red}{V}~N) \rightarrow_{\beta V} (\textcolor{red}{V}~N')
    \end{array}
  \end{displaymath}
\end{block}
\end{frame}
\begin{frame}[fragile]
  \frametitle{Evaluation in Functional Programming Languages}
  \begin{block}{Evaluation of closed terms}
    \begin{itemize}[<+->]
    \item Reduction is restricted to \textbf{closed terms} and stops at weak head-normal forms.
    \item Consequence: substitution need not rename variables!
    \item Substitution can be avoided by implementation tricks
    \item Datatypes are not encoded, but added to the calculus
    \end{itemize}
  \end{block}
\end{frame}

\begin{frame}[fragile]
  \frametitle{Applied Lambda Calculus}
  \begin{block}{Syntax}
    Add constants as values to the pure lambda calculus
    \begin{align*}
      L, M, N &::= x \mid \lambda x.M \mid M\, N \mid C\\
      C    &::= \TRUE \mid \FALSE \mid \IF 
           \mid 0 \mid 1 \mid \dots \mid \SUCC \mid \dots 
           \mid \PAIR \mid \FST \mid \SND \\
      V,W  &::=  \lambda x.M \mid C
           \mid \IF\,V \mid  \IF\,V\,M %\mid +\,V \mid -\,V
           \mid \PAIR\, M \mid \PAIR\, M\, N
    \end{align*}
  \end{block}
  \begin{block}{Semantics (call-by-name)}
    $\beta$ reduction and \textbf{$\delta$ reduction rules} for the constants
    \begin{align*}
      (\lambda x.M)\,N \rightarrow_\beta M[x \mapsto N]
      &&
      \IF\,\TRUE\,M\,N \rightarrow_\delta M
      &&
      \IF\,\FALSE\,M\,N \rightarrow_\delta N
      \\
      &&
      \FST\,(\PAIR\,M\,N) \rightarrow_\delta M
      &&
      \SND\,(\PAIR\,M\,N) \rightarrow_\delta N
    \end{align*}
  \end{block}
\end{frame}             


\begin{frame}[fragile]
  \frametitle{Handling Constants}
  \begin{block}<+->{Context rule for constants}%\VSPBLS
    \begin{displaymath}
      \begin{array}{c}
        N \rightarrow_x N'
        \qquad V \in \{ \IF, \FST, \SND, \SUCC \}
        \\\hline
        (V~N) \rightarrow_x (V~N')
      \end{array}
    \end{displaymath}
  \end{block}
  
  
  \begin{block}<+->{New source of errors: stuck terms}
    \begin{itemize}
    \item Pure lambda calculus: closed term is either value or can reduce
    \item Applied lambda calculus: there are closed non-value terms that cannot be reduced
      \begin{itemize}
      \item $\TRUE\,V$ 
      \item $\IF\,(\lambda x.L)\, M\, N$
      \item $\FST\,(\lambda x.M)$
      \item $\IF\,(\PAIR\, M\, N)\, M'\, N'$
      \item $\dots$
      \end{itemize}
    \item These terms are \textbf{stuck terms}
    \end{itemize}
  \end{block}
\end{frame}

\begin{frame}
  \fontsize{1.5cm}{1.6cm}\selectfont\centering{}%
  Typing
\end{frame}

\begin{frame}[fragile]
  \frametitle{Typing rules out stuck terms }
  \begin{alertblock}<+->{What is a typing?}
    \begin{itemize}
    \item Typing $M:T$ is a relation between terms and types
    \item Typing characterizes terms with a certain behavior
  \end{itemize}
  \end{alertblock}
  \begin{alertblock}<+->{\textbf{Simple types} for the applied lambda calculus}
    \begin{align*}
      T &::= T \to T \mid \Nat \mid \Bool \mid \Pair\,T\,T
    \end{align*}
    \begin{block}{Intended Behavior}
      If $M:T$, then $M$ is not stuck.
    \end{block}
  \end{alertblock}
\end{frame}

\begin{frame}[fragile]
  \frametitle{Simple Types}
  \begin{alertblock}<+->{Definition: Typing assumption (environment)}
    $A ::= \cdot \mid A, x:T$ if $x\notin A$
    \qquad
    specifies typing assumption for variables
  \end{alertblock}
  \begin{alertblock}<+->{Definition: Typing judgment}
    $A \vdash M : T$
    \qquad
    ``under typing assumption $A$, term $M$ has type $T$''
  \end{alertblock}
  \begin{block}<+->{Definition: Typing rules --- lambda calculus with numbers}
    \begin{mathpar}
    \inferrule[Var]{}{\Tenv, x:T, \Tenv' \vdash x : T}

    \inferrule[Lam]{
      \Tenv, x : T \vdash M : T'
    }{
      \Tenv \vdash \Lam{x} M : \Tfun{T}{T'}}
    
    \inferrule[App]{
      \Tenv \vdash M : \Tfun{T}{T'} \\
      \Tenv \vdash N : T
    }{
      \Tenv \vdash \App{M}{N} : T'
    }

    \\
    \inferrule[Num]{}{\Tenv \vdash  n : \Nat}

    \inferrule[Succ]{\Tenv \vdash M : \Nat}{\Tenv \vdash \Succ M :
      \Nat
    }
  \end{mathpar}
  \end{block}
\end{frame}             

\begin{frame}[fragile]
  \frametitle{More typing rules}
  \begin{alertblock}<+->{Definition: Typing rules --- boolean fragment}
    \begin{mathpar}
      \inferrule[TRUE]{}{\Tenv \vdash \TRUE : \Bool}

      \inferrule[FALSE]{}{\Tenv \vdash \FALSE : \Bool}

      \inferrule[IF]{
        \Tenv \vdash L : \Bool \\
        \Tenv \vdash M : T \\
        \Tenv \vdash N : T
      }{\Tenv \vdash \IF\,L\,M\,N : T}
    \end{mathpar}
  \end{alertblock}
\begin{block}<+->{Definition: Typing rules --- pairs}
  \begin{mathpar}
    \inferrule[PAIR]{
      \Tenv \vdash M : T \\
      \Tenv \vdash N : T'
    }{
      \Tenv \vdash \PAIR\,M\,N : \Pair\,T\,T'
    }

    \inferrule[FST]{
      \Tenv \vdash M : \Pair\,T\,T'
    }{
      \Tenv \vdash \FST\,M : T
    }

    \inferrule[SND]{
      \Tenv \vdash M : \Pair\,T\,T'
    }{
      \Tenv \vdash \SND\,M : T'
    }
  \end{mathpar}
\end{block}
\end{frame}

\begin{frame}
  \frametitle{Example Inference Tree}
  \begin{mathpar}
    \infer{
      \infer{
        \infer{
          \infer{}{ \dots \vdash f : \Tfun\alpha\alpha
            \\
            \infer{
              \infer{}{\dots \vdash f : \Tfun\alpha\alpha \\
              \infer{}{\dots \vdash x : \alpha}}
            }{
            \dots \vdash \App{f}x : \alpha}
          }
        }{
          f : \Tfun\alpha\alpha, x : \alpha \vdash \App{f} (\App{f}
          x) : \alpha
        }
      }{
      f : \Tfun\alpha\alpha \vdash \Lam x \App{f} (\App{f} x) : \Tfun\alpha\alpha}
    }{\cdot \vdash \Lam{f}\Lam x \App{f} (\App{f} x) : \Tfun{(\Tfun\alpha\alpha)}{\Tfun\alpha\alpha}}
  \end{mathpar}
\end{frame}

\begin{frame}
  \frametitle{Type Soundness}
  \vspace{-\baselineskip}
  \begin{block}{Type Preservation}
    If $\cdot \vdash M : T$ and $M \rightarrow N$, then $\cdot
    \vdash N : T$.
  \end{block}
  Proof by induction on $M \rightarrow N$.
  \begin{block}{Progress}
    If $\cdot \vdash M : T$, then either $M$ is
    a value or there exists $M'$ such that $M \rightarrow M'$.
  \end{block}
  Proof by induction on $\Tenv \vdash M : T$.
  \begin{block}{Type Soundness}
    If $\cdot\vdash M : T$, then either
    \begin{enumerate}
    \item exists $V$ such that $M \rightarrow^* V$ or
    \item for each $N$, such that $M \rightarrow^* N$ there exists
      $N'$ such that $N\rightarrow N'$.
    \end{enumerate}
  \end{block}
\end{frame}


\begin{frame}[fragile]
  \Huge
  \begin{center}
    {Type Inference for the Simply-Typed Lambda Calculus}
  \end{center}
\end{frame}

\begin{frame}
  \frametitle{Type Inference for the Simply-Typed Lambda Calculus (STLC)}
  \begin{block}{Typing Problems}
    \begin{itemize}
    \item Type checking: Given environment $\Tenv$, a term $M$ and a
      type $T$, is $\Tenv \vdash M : T$ derivable?
    \item Type inference: Given a term $M$, are there $\Tenv$ and $T$ such
      that $\Tenv\vdash M : T$ is derivable?
    \end{itemize}
  \end{block}
  \begin{block}<2->{Typing Problems for STLC}
    \begin{itemize}
    \item Type checking and type inference are decidable for STLC
    \item Moreover, for each typable $M$ there is a \textbf{principal typing}
      $\Tenv\vdash M:T$ such that any other typing is a
      substitution instance of the principal typing.
    \end{itemize}
  \end{block}
\end{frame}


\begin{frame}
  \frametitle{Prerequisites for Type Inference for STLC}
  \framesubtitle{Unification}
  Let $\calE$ be a set of equations on types.
  \begin{block}{Unifiers and Most General Unifiers}
    \begin{itemize}
    \item A substitution $S$ is a \textbf{unifier of $\calE$} if,
      for each $T \doteq T'\in \calE$, it holds that $ST
      = ST'$.
    \item 
      A substitution $S$ is a \textbf{most general unifier of
        $\calE$} if $S$ is a unifier of $\calE$ and for every
      other unifier $S'$ of $\calE$, there is a substitution
      $U$ such that $S' = U \circ S$.
    \end{itemize}
  \end{block}
  \begin{block}<2->{Unification}
    There is an algorithm $\calU$ that, on input $\calE$, either returns a
    most general unifier of $\calE$ or fails if none exists.
  \end{block}
\end{frame}

\begin{frame}
  \frametitle{Principal Type Inference for STLC}
  %\vspace{-\baselineskip}
  \small
  The algorithm (due to John Mitchell) transforms a term into a principal typing judgment for
  the term or fails if no typing exists.
  \begin{displaymath}
    \begin{array}{lcl}
      \calP (x) &=& \mathbf{return}\ \Encode{x:\alpha \vdash x : \alpha} \\
      \calP (\Lam{x} M) &=&
      \mathbf{let}\ \Encode{\Tenv \vdash M : T} \gets \calP (M) \ \mathbf{in}\\
      && \mathbf{if}\ \Tenv\ \mathbf{has\ form}\ \Tenv', x:T_x \ \mathbf{then\ return}\ \Encode{\Tenv'
      \vdash \Lam{x}M : T_x \to T} \\
      && \mathbf{else}\ \mathbf{choose}\ \alpha\notin\var{(\Tenv,T)}\
      \mathbf{in} \\
      &&\qquad \mathbf{return}\ \Encode{\Tenv \vdash \Lam{x}M : \alpha\to T} \\
      \calP (\App{M_0}{M_1}) &=&
      \mathbf{let}\ \Encode{\Tenv_0\vdash M_0 : T_0} \gets \calP (M_0)\ \mathbf{in} \\
      && \mathbf{let}\ \Encode{\Tenv_1\vdash M_1 : T_1} \gets \calP (M_1)\
      \mathbf{in} \\
      && \mathbf{with\ disjoint\ type\ variables\ in\ } (\Tenv_0,
      T_0)\ \mathbf{and}\ (\Tenv_1, T_1) \\
      && \mathbf{choose}\ \alpha\notin \var{(\Tenv_0,\Tenv_1,
        T_0,T_1)}\ \mathbf{in} \\
      && \mathbf{let}\ S \gets \calU (\Tenv_0 \doteq \Tenv_1, T_0
      \doteq T_1 \to \alpha)\
      \mathbf{in} \\
      && \mathbf{return}\ \Encode{S\Tenv_0 \cup S\Tenv_1 \vdash
      \App{M_0}{M_1} : S\alpha} \\
      \calP (n) &=& \mathbf{return}\ \Encode{\cdot \vdash 
      n : \Nat} \\
      \calP (\Succ M) &=& \mathbf{let}\ \Encode{\Tenv \vdash M : T} \gets
      \calP (M)\ \mathbf{in} \\
      && \mathbf{let}\ S \gets \calU (T\doteq\Nat)\ \mathbf{in} \\\
      && \mathbf{return}\ \Encode{S\Tenv \vdash \Succ M: \Nat} 
    \end{array}
  \end{displaymath}
\end{frame}

\begin{frame}
  \frametitle{Properties of Type Inference}
  \begin{block}{Soundness}
    If $\calP (M) = \Encode{\Tenv \vdash M: T}$, then $\Tenv \vdash M: T$ is derivable.
  \end{block}
  \begin{block}{Completeness}
    If $\Tenv \vdash M : T$ is derivable, then $\calP (M)$ succeeds with result $\Encode{\Tenv' \vdash M: T'}$ such that $\Tenv = S\Tenv'$ and $T = S T'$ for some substitution $S$. 
  \end{block}
\end{frame}

\begin{frame}
  \frametitle{Wrapup}
  \begin{itemize}
  \item Call-by-name and call-by-value are deterministic evaluation strategies that are more efficient than full $\beta$ reduction
  \item Applied lambda calculus contains constants that encode operations on datatypes
  \item Applied lambda calculus can have stuck terms
  \item Simple types avoid stuck terms
  \item Type checking and type inference for simple types is decidable
  \item There is a sound and complete algorithm for type inference for simple types
  \end{itemize}
\end{frame}
\end{document}


